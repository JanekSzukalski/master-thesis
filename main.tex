%%\RequirePackage{fix-cm} %%Dla nietypowych rozmiarów czcionki do umieszzczenia przed \documentclass
%%\fontsize{<fontsize>}{<baselineskip>}\selectfont 

\documentclass[a4paper,12pt,twoside,polish]{book}
\usepackage[a4paper,left=30mm,right=25mm,top=25mm,bottom=25mm,includefoot=false,includehead=false]{geometry}

\usepackage{amsmath,amsthm,amssymb}
\usepackage{enumerate}
\usepackage{color}


\usepackage{pdfpages}%dodawanie do dokumentu plików pdf
\usepackage{graphicx}
\usepackage[export]{adjustbox} %%do pozycjonowania plików graficznych
%%Opcja pdftex dla plików z rozszerzeniami png, jpg, pdf, tif, mps -  pdflatex
%% domyślnie jest wprowadzona opcja dvips dla plików EPS - kompliacja do DVI i PS
\usepackage{makeidx}
\makeindex


\usepackage[MeX]{polski}
\usepackage[polish]{babel}
\usepackage[utf8]{inputenc} %w przypadku kodowania UTF-8

%albo
%usepackage[cp1250]{inputenc} %w przypadku kodowania windows-1250

\usepackage{tkz-graph}
\usetikzlibrary{arrows}
\usetikzlibrary{arrows.meta}
\usetikzlibrary{calc}
\usetikzlibrary{shadows}
\usetikzlibrary{external}
\usepackage{circuitikz}
\usepackage{pgfplots}
\pgfplotsset{compat=1.18}
%%\usepackage{hyperref}

%%dodanie kropek po tytułach sekcji (section) w spisie treści; w stylu article nowe, inny odstęp kropek
\usepackage{tocloft}
\renewcommand\cftchapaftersnum{.}% 
\renewcommand\cftchapdotsep{\cftdotsep}%

%%Interlinia. Zgodnie z Rozporzadzeniem możliwe ustawienia od 1.0 do 1.5
%%Dobra opcja to 1.2 
\renewcommand{\baselinestretch}{1.2} %%możliwe od 1.0 do 1.5

%%%THEOREMS AND ENVIROS%%%
\theoremstyle{definition}
\newtheorem{definition}{Definicja}[section]
\newtheorem{problem}{Problem}[section]

\theoremstyle{plain}
\newtheorem{theorem}{Twierdzenie}[section]
\newtheorem{lemma}{Lemat}[section]
\newtheorem{corollary}{Wniosek}[section]
\newtheorem{remark}{Uwaga}[section]
\newtheorem{observation}{Obserwacja}[section]

\theoremstyle{definition}
\newtheorem{example}{Przykład}[section]


% \newtheorem{theorem}{\small\bf Twierdzenie}[chapter]
% \newtheorem{lemma}[twierdzenie]{\small\bf Lemat}
% \theoremstyle{remark}
% \newtheorem{remark}[twierdzenie]{\small\bf Uwaga}

% \theoremstyle{definition}
% \newtheorem{definition}[twierdzenie]{\small\bf Definicja}
% \newtheorem{corollary}[twierdzenie]{\small\bf Wniosek}
% \newtheorem{example}[twierdzenie]{\small\bf Przykład}

\newcommand{\C}{\mathbb{C}}
\newcommand{\R}{\mathbb{R}}
\newcommand{\Q}{\mathbb{Q}}
\newcommand{\Z}{\mathbb{Z}}
\newcommand{\N}{\mathbb{N}}
\renewcommand{\Re}{\operatorname{Re}}
\newcommand{\Sp}{\textnormal{Sp}}
\newcommand{\nn}{\nonumber}
\newcommand{\Rn}{\mathbb{R}^n}
\newcommand{\f}{\varphi}
\newcommand{\be}{\begin{eqnarray}}
\newcommand{\ee}{\end{eqnarray}}
\newcommand{\bd}{\begin{eqnarray*}}
\newcommand{\ed}{\end{eqnarray*}}

\DeclareMathOperator{\Ker}{Ker}
\DeclareMathOperator{\Imm}{Im}
\DeclareMathOperator{\Dom}{Dom}
\DeclareMathOperator{\tr}{tr}


%%%%%%%%%%%%%%%%%%%%%%%%%%%%%%%%%%%%%%%%%%%%%%%%%%%%%%%%%%%%%%%%%%%%%%
%%%%%%%%%%%%%%%%%%%%%%%%%%%%%%%%%%%%%%%%%%%%%%%%%%%%%%%%%%%%%%%%%%%%%%
%%% Poniżej dostosowanie stylu book tak aby każdy rozdział
%%% zaczynał się od nieparzystej strony. To powoduje wstawienie 
%%% na poprzedniej stronie kolejnej pustej parzystej strony
%%% z atrybutami poprzedniego rozdziału. Niżej dodatkowo
%%% dostosowanie stylu, które kasuje tekst z nagłówka
%%% i stopki poprzedniej strony.
%%%%%%%%%%%%%%%%%%%%%%%%%%%%%%%%%%%%%%%%
%%%%%%%%%%%%%%%%%%%%%%%%%%%%%%%%%%%%%%%%%%%%%%%%%%%%%%%%%%%%%%%%%%%%%%%%%%%%%%%%

\setlength{\headheight}{13.6pt}
\usepackage{fancyhdr} %% aby zmienić nagłówek i stopkę
\pagestyle{fancy}
\fancyhead{}
\fancyfoot{}
\renewcommand{\chaptermark}[1]{\markboth{\thechapter\ #1}{}}
\renewcommand{\sectionmark}[1]{\markright{\thesection\ #1}}
\fancyfoot[LE,RO]{\normalfont  \thepage}
\fancyhead[RO]{\normalfont \small \itshape \rightmark}
\fancyhead[LE]{\normalfont \small \itshape \leftmark}
\fancypagestyle{plain}{\fancyhead{}\renewcommand{\headrulewidth}{0pt}}
\makeatletter
\def\clearpage{\ifvmode\ifnum\@dbltopnum=\m@ne\ifdim\pagetotal<\topskip\hbox{}\fi\fi\fi\newpage\write\m@ne{}\vbox{}\penalty-\@Mi\thispagestyle{empty}}
\makeatother

%%%%%%%%%%%%%%%%%%%%%%%%%%%%%%%%%%%%%%%%%%%%%%%%%%%%%%%%%%%%%%%%%%%%%%%%%%%%%%%%%%%%%%%%%%%%%%%%%%%%%%%%%%%%%%%%%%%%%%%%%%%%%%

\usepackage[figurename=Rys.]{caption}
\DeclareCaptionFormat{custom}
{\textbf{#1#2}{\small #3}}
\captionsetup{format=custom}


\begin{document}

\includepdf{moja_strona_tytulowa.pdf}%%wczytanie strony tytułowej, którą trzeba zapisać w formacie pdf

\newpage
\thispagestyle{empty} 
{\color{white}-}
%%%%%%%%%%%%%%%%%%%%%%%%%%%%%%%%%%%%%%%%%
\newpage
\thispagestyle{empty}
\tableofcontents

%%%%%%%%%%%%%%%%%%%%%%%%%%%%%%%%%%%%%%%%%
\chapter*{  }
\addcontentsline{toc}{chapter}{Streszczenie}
\addcontentsline{toc}{chapter}{Słowa kluczowe}
\begin{center}
\begin{normalsize}
{\textbf{Streszczenie}}
\end{normalsize} %%Streszczenie jest wymagane
\end{center}
\begin{quotation} %%maksymalnie 2000 znaków razem ze spacjami
\noindent Tekst streszczenia pracy dyplomowej. 
\end{quotation}
\vspace{3.5cm}
\begin{center}
\begin{normalsize}{\textbf{Słowa kluczowe}}\end{normalsize} %%Słowa kluczowe w języku polskim są wymagane
\end{center}
\begin{quotation}
\noindent Słowa kluczowe, słowa kluczowe, słowa kluczowe, słowa kluczowe.%%do pięciu słów
\end{quotation}
\vspace{3.5cm}
\begin{center}
\begin{normalsize}{\textbf{Keywords}}\end{normalsize} %%Słowa kluczowe w języku angielskim. 
\end{center}
\begin{quotation}
\noindent Słowa kluczowe w języku angielskim.%%do pięciu słów
\end{quotation}
\clearpage


\chapter*{Wstęp}\label{wstep} %%Wymagany
\markboth{Wstęp}{Wstęp}
\addcontentsline{toc}{chapter}{Wstęp}
\     %razem z pustą linią niżej powoduje akapit (tutaj na początku rozdziału Wstęp)

,,We wstępie należy zarysować ogólne tło tematu pracy (badanego problemu, projektu), wskazać przesłanki wyboru tematu pracy, określić problematykę" (Z Zarządzenia Nr 75/2022 Rektora Politechniki Łódzkiej z 22 grudnia 2022 r., Załącznik nr 10).









\newpage
Druga strona wstępu. 

\newpage
Trzecia strona wstępu.

\chapter*{Cel i zakres pracy} %%Wymagane
\markboth{Cel i zakres pracy}{Cel i zakres pracy}
\addcontentsline{toc}{chapter}{Cel i zakres pracy}
\     %razem z pustą linią niżej powoduje akapit (tutaj na początku rozdziału Wstęp)

Zgodnie z Załącznikiem nr 10 do Regulaminu dyplomowania w Politechnice Łódzkiej do Zarządzenia Nr 75/2022 Rektora Politechniki Łódzkiej z 22 grudnia 2022 r. część główna pracy dyplomowej musi zawierać cel i zakres pracy. Należy przedstawić to co było oczekiwanym rezultatem prowadzonych badań oraz zaprezentować zakres pracy, badania, działania jakie trzeba było wykonać aby postawiony cel osiągnąć. 

\newpage
Cel i zakres pracy -- druga strona.


\chapter{Preliminaria}\label{pre}


\section{Sekcja 1}
  


\chapter{Tytuł rozdziału drugiego}

\section{Sekcja 1}

\section{Sekcja 2}

\section{Zapisywanie kodu źródłowego procedur}

Kody źródłowe procedur lub funkcji zapisanych na przykład w systemie R lub Matlab można zapisać w systemie {\LaTeX} wykorzystując środowisko \verb+verbatim+. Na przykład:
\begin{verbatim}
function[wynik]=SumaKwadratow(N)
%Suma kwadratów liczb 1,2,...,N.

wynik=0;
odd=-1;
term=0;

for i=1:N
   odd=odd+2;
   term=term+odd;
   wynik=wynik+term;
endfor
endfunction
\end{verbatim}

\noindent Ewentualnie można zastosować tekst maszynowy korzystając z polecenia \verb+\texttt+.



\section{Rysunki}
\begin{figure}[!ht]
    \centering
\begin{tikzpicture}[scale=0.7]
\draw [->](0,-5)--(0,2);
\draw [->](0,0)--(11,0);
\draw (10.9,-0.5) node {$n$};
\draw (-.7,2) node {$w$};
\draw (10.6,2) node {\footnotesize{$X^u$}};
\draw (9.5,-0.4) node {\footnotesize{$1$}};
\draw [fill] (0,0) circle (3pt);
\draw [fill] (9,0) circle (3pt);
\draw (9,0)--(9,-4);
\draw (9,-4)--(0,0);
\draw (10.5,-2) node {\footnotesize{Bok I}};
\draw (4.5,0.7) node {\footnotesize{Bok II}};
\draw (3.5,-2.5) node {\footnotesize{Bok III}};
\draw [->] (6,0.5) -- (6,-0.5);
\draw [->] (9.5,-1) -- (8.5,-2);
\draw [->] (6.5,-3.3) -- (5.7,-2.2);
\draw  [black] (0,0) to  [out=-18,in=242] node[currarrow, pos=0.5, xscale=-1, sloped, scale=1] {} (9,0);
\draw [dashed] (10,2)--(6.5,-5);
\end{tikzpicture}
    \caption{Zbiór dodatnio niezmienniczy modelu wraz z zaznaczoną orbitą heterokliniczną.}
    \label{rysmodelfktrojkat}
\end{figure}

%Dodawanie pliku graficznego
\begin{figure}
\centering
\includegraphics[height=5.8cm]{rozm 6.jpg}
\caption{Portret fazowy układu z zaznaczonymi podprzestrzeniami stycznymi $X^s$ oraz $X^u$ oraz rozmaitością stabilną i niestabilną.}
\label{rys rozm 6}
\end{figure}

Rysunek \ref{rysmodelfktrojkat} pokazuje zbiór dodatnio niezmienniczy rozważanego modelu wraz z~zaznaczoną orbitą heterokliniczną.

Wiele przykładów rysunków tworzonych z wykorzystaniem systemu {\LaTeX} można odszukać na przykład w książce \cite{Goossens-Rahtz-Mittelbach}.

\section{Cytowania}

Spis literatury wykorzystywanej w pracy jest obowiązkowym elementem. Spisie literatury umieszczamy tylko te pozycje, do których odnosimy się w pracy. W miejscu odniesienia się do umieszczamy symbol jakim oznaczona jest dana pozycja w spisie literatury.

Do poszczególnych pozycji ze spisu literatury odnosimy się za pomocą polecenia \verb+\cite+. 

Na przykład polecenie \verb+\cite{Banasiak-Tchamga}+ generuje w pliku pdf następujący symbol cytowania \cite{Banasiak-Tchamga}.

Zalecany jest jednolity styl typu autor--data odwoływania się do źródeł. Informacje na temat stylu typu autor--data są zamieszczone na stronach Biblioteki Politechniki Łódzkiej pod adresem
\bd
\text{http://bg.p.lodz.pl/bibliografia-zalacznikowa}
\ed



%%\newpage
\section{Tabele}

\begin{table}[!ht]
\centering
\begin{tabular}{|c|c|c|}

\hline
   Parametr & Wartość & Jednostka \\
   \hline \hline
   $r_{1}$ & 0.514 & $\frac{1}{\text{dni}}$ \\
   \hline
   $r_{2}$ & 0.18 & $\frac{1}{\text{dni}}$ \\
   \hline
   $K_{1}$ & 9.8039 $\cdot$ 10$^{8}$ & komórki \\
   \hline
   $K_{2}$ & 10$^{9}$ & komórki \\
   \hline
   $\delta$ & 0.2 & $\frac{\text{komórka}}{\text{dni} \cdot \text{pg/ml}}$\\
   \hline
   $m$ & 10$^{5}$ & komórki \\
   \hline
   $a$ & 0.1 & $\frac{1}{\text{dni}}$\\
   \hline
   $r$ & 9.6 $\cdot$ 10$^{3}$ & $\frac{\text{komórka}}{\text{dni}}$\\
   \hline
   $k$ & 5 $\cdot$ 10$^{-11}$ & $\frac{1}{\text{dzień komórki}}$\\
   \hline
   $c_{1}$ & 10$^{-7}$ & $\frac{1}{\text{dzień komórki}}$\\
   \hline
   $\beta$ & 0.09 & $\frac{\text{komórka}}{\text{dni} \cdot \text{pg/ml}}$\\
   \hline
   $\eta$ & 10$^{3}$ & komórki\\
   \hline
   $\nu$ & 10$^{-3}$ & $\frac{1}{\text{liczba cząsteczek}}$\\
   \hline
   $\alpha$ & 9 & $\frac{\text{pg/ml}}{\text{dni} \cdot \text{komórka}^{2}}$\\
   \hline
   $b$ & 10$^{-3}$ & $\frac{1}{\text{liczba cząsteczek}}$\\
   \hline
   $\mu$ & 34 & $\frac{1}{\text{dni}}$\\
   \hline
   $c$ & 5000 & $\frac{\text{liczba cząsteczek}}{\text{dni} \cdot \text{komórki}}$\\
   \hline
   $d$ & 8.3178 & $\frac{1}{\text{dni}}$\\
   \hline
   $d_{1}$ & 1.1 $\cdot$ 10$^{-10}$ & $\frac{\text{komórka}}{\text{dni}}$\\
   \hline
   $d_{2}$ & 4.8 $\cdot$ 10$^{-10}$ & $\frac{\text{komórka}}{\text{dni}}$\\
   \hline
\end{tabular}
\caption{ Wartości parametrów modelu.}
\label{tabela_2}
\end{table}

W tabeli \ref{tabela_2} przedstawiono wartości parametrów omawianych w pracy.


\subsection{Wyliczenia}

Wyliczenia powinny mieć w stałej pracy taki sam styl (na przykład z użyciem kropek albo myślników). wyliczenia tworzywmy w środowisku \verb+itemize+. Na przykład:

\begin{enumerate}
\item Tekst w punkcie pierwszym.
\item Tekst w punkcie drugim:
        \begin{itemize}
        \item pierwszy podpunkt punktu drugiego,
        \item drugi podpunkt punktu drugiego,
        \item trzeci podpunkt punktu drugiego.
        \end{itemize}
\item Tekst w punkcie trzecim:
        \begin{itemize}
        \item pierwszy podpunkt punktu trzeciego,
        \item drugi podpunkt punktu trzeciego.
        \end{itemize}
\end{enumerate}

\noindent Inny przykład z wykorzystaniem tylko numeracji:

\begin{enumerate}
\item Punkt pierwszy:
   \begin{enumerate}
   \item podpunkt pierwszy,
   \item podpunkt drugi.
   \end{enumerate}
\item Punkt 2.
\item Punkt 3.
\begin{enumerate}
   \item podpunkt pierwszy punktu trzeciego,
   \item podpunkt drugi punktu trzeciego.
   \end{enumerate}
\item Punkt 4.
\end{enumerate}

Można również zdefiniować własną numerację. Na przykład:

\begin{enumerate}
\item[a$_{1}$.] Punkt pierwszy.
\item[a$_{2}$.] Punkt drugi.
\end{enumerate}

\chapter*{Podsumowanie} %%Podsumowanie jest wymagane
\markboth{Podsumowanie}{Podsumowanie}
\addcontentsline{toc}{chapter}{Podsumowanie} %%Dodanie tytułu rozdziału Podsumowanie do spisu treści

Część główna pracy powinna zawierać podsumowanie -- zawierające syntezę wniosków opartą na udowodnionych przesłankach i podsumowanie wyników podjętego zagadnienia/rozpoznania badawczego.

Należy dodać informację o tym co dyplomant wykonał samodzielnie, jakie nowe rezultaty lub interpretacje zostały uzyskane.

\newpage
Druga strona podsumowania.


\newpage
\renewcommand{\bibname}{Literatura} %%Spis literatury jest wymagany
\addcontentsline{toc}{chapter}{Literatura} %%Dodanie sekcji Literatura do spisu treści
\begin{thebibliography}{}

\end{thebibliography}



\newpage
\clearpage
\thispagestyle{empty}



\addcontentsline{toc}{chapter}{Spis rysunków} %%Dodanie sekcji Spis rysunków do spisu treści

\listoffigures  %%Spis rysunków nie jest wymagany

\bigskip
\bigskip
\bigskip

\addcontentsline{toc}{chapter}{Spis tabel} %%Dodanie sekcji Spis tabel do spisu treści
\listoftables %%Spis tabel nie jest wymagany

\clearpage
\newpage

{\color{white}-}


{\color{white}-}


{\color{white}-}



%%Poniżej przykład spisu symboli z ich opisem. Nie jest wymagany.
{\noindent{\huge{\textbf{Spis symboli i oznaczeń}}}}
%%Spis symboli i oznaczeń nie jest wymagany
\addcontentsline{toc}{chapter}{Spis symboli i oznaczeń}
{\color{white}-}


{\color{white}-}


{\color{white}-}


\begin{tabular}{cp{0.8\textwidth}}
$X^{\mathbb{N}}$ &  Zbi\'{o}r ciągów o wartościach w X \\
$L_{m}^{n}(\mathbb{R})$ &  Zbi\'{o}r wszystkich macierzy o $m$ wierszach i $n$ kolumnach o warto\'{s}ciach w pierścieniu liczb rzeczywistych \\

\end{tabular}



\newpage
\addcontentsline{toc}{chapter}{Skorowidz}
\printindex %%Skorowidz nie jest wymagany

\end{document}
